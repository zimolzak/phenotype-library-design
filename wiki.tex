\documentclass{tufte-handout}
\title{Recap of MAVERIC Wiki Changes}
\author{Andy Zimolzak, MD, MMSc}
\date{November 19, 2014}

\begin{document}

\maketitle

~\\

The MAVERIC wiki\footnote{http://vhabosapp8/wiki} was created in
December 2010. It was evidently used a fair amount through October
2011. Starting June 2014, usage has picked up again. The
\texttt{vhabosapp8} server that hosts the wiki also hosts Subversion
repositories and the Jira issue and project tracker. Currently 13 out
of 74 GB of space is free on \texttt{vhabosapp8}.

\section{Problem/niche}

The Precision Oncology Project and the Point of Care Trial project
rely on a workflow that goes as follows. \emph{First,} subject matter
expert knows that we want to track the effect of drug $x$ on (for
example) serum sodium. SME attempts to operationalize the concept of
serum sodium, suggesting a list of terms to be searched for in a field
(e.g.\ \texttt{LabChemTestName}) of CDW. \emph{Second,} data pull team
returns a list of lab tests matching these search
terms.\footnote{probably generating SQL code in the process.}
\emph{Third,} SME reviews list of tests and decides (adjudicates)
which items truly match the concept. SME returns either a refined
general rule to specify which items match the concept, or an
enumerated list of which items match the concept. \emph{Fourth,} data
pull team creates query that represents this
adjudication.\footnote{probably generating refined SQL code.} The
clinical SME may give formal approval or ``sign off'' on the data pull
at step 3, or on the results of step 4.

Data pull and NLP teams intend to use the wiki to capture this
workflow so that these deliverables are \emph{available for reuse} in
a library, for the purposes of other projects; \emph{available for
  discussion} with their ``provenance'' (responsible SME and data
extractor) clearly noted; and \emph{available for renewal,} with their
date of creation clearly noted, because CDW elements may change over
time. The rationale for the SME including/excluding certain tests can
be documented. 

\section{Questions for the group}

What should we call the deliverable that the data pull expert and the
SME create in collaboration?\footnote{Suggestions: data model, concept
model, algorithm, adjudication, library}

Is the wiki the right place for SQL code?

Training on the wiki?

How does wiki fit in with data pull request tracker?

% left off here

\section{Themes}

\begin{itemize}
\item Data Access and Secure Server Development
\item Model/algorithm development

\begin{itemize}
\item Communication
\item Documenting the process
\item Storing results
\end{itemize}

\item Model/algorithm management

\begin{itemize}
\item Storing metadata (DMR)
\item Storing models themselves
\item Model test/validation process (who signs off, on what?)
\item Storing model test and validation results
\end{itemize}
\end{itemize}

Among others, this encompasses the wiki project as well as the data
pull request tracker. These projects originated as solutions to
existing problems, and both are hopefully approaching full roll-out
phase. As such, it may be important to have defined training
objectives and roll-out plans for each tool, so that they become
integrated into workflows, rather than remain as offshoot projects.
 
Next Wednesday's meeting will be abbreviated due to the CSPCC monthly
speaker event occurring at noon (topic is fraud in medical research,
should be very interesting), so we will only have time to talk at
length about one topic.
 
Andy and Ned, if you could have a recap of the changes that have
happened with the wiki project for next week, as well as your plans
for next steps with the project, I think that would be of general
interest to the members of this meeting.
 
There has also been some feedback regarding the name of this meeting,
since it does not directly pertain to ``data.'' One name proposal is
``Model/Algorithm environments meeting'' but I don't know if that is
too vague. Thoughts?

\section{About MediaWiki}

``A wiki is a Web application which allows people to add, modify, or
delete content in collaboration with
others.''\footnote{http://en.wikipedia.org/wiki/Wiki} MediaWiki is the
software that makes Wikipedia run, but anyone can install it to their
Web server to create their own wiki. The \texttt{vhabosapp8} server
runs Red Hat Enterprise Linux, Apache, MySQL, and PHP.

\section{What has changed recently}

\paragraph{October 23--28} I created several standard templates that
can be used in multiple places, to avoid duplicating the same wikitext
code. These are mainly for Data Pull and NLP subsections of the wiki. 

\paragraph{October 29} Organize the Data Pull wiki page structure. 

\paragraph{October 30} Start using for ``POP Additional Documents''
and sodium.

\paragraph{November 5} Shadow wiki created.

\paragraph{November 6} Magnesium, ALT.

\paragraph{November 12} Shadow wiki becomes real wiki.

\paragraph{November 12--14} Magnesium, ALT, POP surgical procedures,
ALT, AST.

\section{Suggestions for you all in future}

Don't upload enormous things. Use talk pages. 

\section{Flaws in wiki}

Where does the actual SQL go? Here is what the current Data Pull
template looks like.

\end{document}

% LocalWords:  MAVERIC vhabosapp Jira workflow SME operationalize CDW DMR CSPCC
% LocalWords:  LabChemTestName workflows MediaWiki MySQL PHP wikitext AST
