\documentclass{tufte-handout}
\title{Recap of MAVERIC Wiki Changes}
\author{Andy Zimolzak, MD, MMSc}
\date{November 19, 2014}

\begin{document}

\maketitle

~\\

The MAVERIC wiki\footnote{http://vhabosapp8/wiki} was created in
December 2010. It was evidently used a fair amount through October
2011\footnote{It was used for things like GenISIS meeting notes,
  documenting GenISIS \& HPC server configurations, and documenting
  server setup procedures. Top contributors were Phsieh,
  Lalitha.viswanath, and Mmcduffie.}, but with virtually no edits in
2012--2013. Starting May 2014, usage has picked up again.\footnote{Now
  used for DCP, POP, data pull, ``phenotype models,'' and NLP. Top
  contributors: Rpourali, Nilla, Beth, Andy Z. Now there are 54
  content pages, and 11 users who have been active in the last 30
  days.}

\section{Problem/niche}

The Precision Oncology Project and the Point of Care Trial project
rely in part on a workflow that goes roughly as follows. \emph{First,}
the subject matter expert knows that we want to track the effect of
drug $x$ on (for example) serum sodium. The SME attempts to
operationalize the concept of serum sodium, suggesting a list of terms
to be searched for in a field (e.g.\ \texttt{LabChemTestName}) of CDW.
\emph{Second,} the data pull team returns a list of lab tests matching
these search terms.\footnote{The list sent to the SME is often an
  Excel file. Data extractor probably generates SQL code in the
  process.} \emph{Third,} the SME reviews the list of tests and
decides (adjudicates) which items truly match the concept. The SME
returns either a refined general rule to specify which items match the
concept, or an enumerated list of which items match the
concept.\footnote{So far I return this adjudication to Beth in the
  form of an edited Excel file.} \emph{Fourth,} the data pull team
creates a query that represents this adjudication.\footnote{probably
  generating refined SQL code.} The clinical SME may give formal
approval or ``sign off'' on the data pull at step 3, or possibly on
the results of step 4.

The data pull and NLP teams intend to use the wiki to capture this
workflow so that these deliverables are \emph{available for reuse} in
a library, for the purposes of other projects; \emph{available for
  discussion} with their ``provenance'' (responsible SME and data
extractor) clearly noted; and \emph{available for renewal,} with their
date of creation clearly noted, because CDW elements may change over
time. The rationale for the SME including/excluding certain tests can
also be documented.\footnote{The basic idea is that the methods of
  science should be documented well enough to be reproducible---in
  other words, ``show your work.'' For our type of science, \emph{code
    is method.} See also http://sciencecodemanifesto.org}

\section{Recent wiki changes}

\paragraph{October 23--29} I created several standard templates that
can be used in multiple places, to avoid duplicating the same wikitext
code. These are mainly for Data Pull and NLP subsections of the wiki.
Also organized the structure of some wiki pages.

\paragraph{October 30} First actual use of the wiki to store
adjudications---specifically to store ``POP Additional Documents''
(cancer-related radiology notes) and sodium.

\paragraph{November 5} Shadow wiki created for testing, using 
MediaWiki version 1.23.6 instead of 1.16.0. This will allow us to use
new MediaWiki extensions to do two things: write wikicode templates
that use conditional logic,\footnote{An example of invoking such a
  template, used on the NLP dashboard page, looks like this:
  \texttt{\{\{Phase|Requirements\}\}}. This expands to a nice looking
  table with one cell highlighted. Elsewhere this template is invoked
  as \texttt{\{\{Phase|Development\}\}}, which expands to the same
  table with a different cell highlighted. This avoids a \emph{lot} of
  copy-pasting of wikicode tables.} and do syntax highlighting of
source code snippets. Ned and I are also given shell accounts on
\texttt{vhabosapp8} thanks to Paul.

\paragraph{November 6--10} Magnesium and ALT lab adjudications uploaded.

\paragraph{November 12} Shadow wiki becomes real wiki.

\paragraph{November 12--14} POP surgical procedures and AST
adjudications uploaded.

\section{Suggestions for you all in future}

Look at the wiki.\footnote{Reminder! http://vhabosapp8/wiki} Create a
user account. Do edit things (be bold). Ask me if you need
help.\footnote{Andrew.Zimolzak@va.gov} Don't upload enormous things.
Use talk pages.

\section{Questions for the group}

\begin{itemize}

\item What should we call the deliverable that the data pull expert
  and the SME create in collaboration?\footnote{Thoughts: data
    model, concept model, algorithm, adjudication, library, phenotype,
    phenotype model.}

\item Is the wiki the right place for SQL code? Or should the wiki
  merely have a pointer to a source control repository?

\item What training on the wiki do you want?

\item How does wiki fit in with data pull request tracker?

\item Should SMEs deliver general rules or enumerations?

\item What should be captured about each algorithm?\footnote{Currently
  caputured on basic algorithm pages on wiki: algorithm version, SQL
  statement, SME name, approval date. On the more detailed wiki form
  we have the following fields: lab name, sample source (body site),
  date created, data source, SQL search terms used, clinical
  adjudication list (link to Excel file), adjudicator name, SQL data
  extractor name, extracted data location, content of extracted data,
  extraction verified by:, SQL extraction code.}

\end{itemize}

\section{Further reading about MediaWiki and our server}

``A wiki is a Web application which allows people to add, modify, or
delete content in collaboration with
others.''\footnote{http://en.wikipedia.org/wiki/Wiki} MediaWiki is the
software that makes Wikipedia run, but anyone can install it to their
Web server to create their own wiki. MediaWiki uses its own markup
language called wikitext.

The \texttt{vhabosapp8} server runs Red Hat Enterprise Linux, Apache,
MySQL, and PHP. It also hosts Subversion repositories and the Jira
issue and project tracker. Currently 13 out of 74 GB of space is free
on \texttt{vhabosapp8}.

\end{document}

% LocalWords:  MAVERIC vhabosapp Jira workflow SME operationalize CDW DMR CSPCC
% LocalWords:  LabChemTestName workflows MediaWiki MySQL PHP wikitext AST HPC
% LocalWords:  GenISIS Phsieh Lalitha viswanath Mmcduffie DCP Rpourali Nilla
% LocalWords:  wikicode
