\documentclass{tufte-handout}
\title{Design of a VA Phenotype Library}
\author{Andy Zimolzak, MD, MMSc}
\date{February 8, 2017}

\begin{document}

\maketitle

Motivation: (a) VA data elements are not harmonized across stations,
and that (b) phenotypes do this harmonization and also build low level
phenotypes into higher order phenotypes.\footnote{For example, 130
  stations ICD codes + labs + ??? $\to$ harmonized ICD + labs + ???
  $\to$ diabetes + neuropathy $\to$ diabetic nephropathy $\to$ which
  is a subset of diabetes with any microvascular complication $\to$
  subset of diabetes with any complication at all, etc.}

\section{What I imagine for a phenotype library}

When a subject matter expert (SME), analyst, statistician, and/or data
pull engineer create a phenotype/adjudication, the result can be SQL
code that represents the SME's rules, or a public database table that
represents the enumerated list of included/excluded data items, or
even a statistical model of the likelihood of a patient having the
given phenotype. Ideally, a robust phenotype library should exist,
serving several functions. (1) The library would make the products of
each SME/data review process \emph{available for viewing} by study
coordinators or others, with ``provenance'' (responsible SME and data
extractor) clearly noted, to allow study staff to trust a given
phenotype. Make it easy to browse an overview of all documents
without having to drill down into 99 different subfolders. (2) The library would make phenotypes \emph{available for
  reuse}, for the purposes of other projects. (3) The library would
allow easy searches across phenotypes, to allow new studies to
discover what already exists. Very important
to search within document content. (4) The library would make phenotypes
\emph{available for renewal,} with their date of creation clearly
noted, because data warehouse elements may change over time. (5) The
library would document the rationale for decisions made by SME and
data pull engineers. Allow collaborative editing of a group of documents. Audit/track all edits. Possibly
link among documents.  
(6) Should be available across VA nationally.

\section{Operationalizing the library}

I think \emph{some form} of wiki serves several of the functions
well,\footnote{Specifically: view provenance, search, and view
  documentation or rationale. All of these functions involve basically
  human-readable items with English sentences.} but I'm not sure
whether the current wiki is the best long-term form of wiki. It has a
couple issues and is not perfect. SharePoint would work OK, except for
the fact SharePoint's search feature does not work at all for us.

A wiki does \emph{not} serve some of the functions as
well.\footnote{reuse phenotype, renew or modify phenotype. these are
  more oriented toward machine-readable items. For example: enumerated
  list of SIDs in a database table like MAVIN.DimLab, SQL code that
  describes a rules-based phenotype, SAS or R or whatever code that
  describes a statistical model of a phenotype.} In my imagined ideal
world, people would use Git software to manage this code. There are
other tools for source code management too.\footnote{e.g.\ IBM
  Rational Team Concert} The main drawback is the learning curve, so I
don't know whether this is actually going to happen. SQL/SAS code on
the wiki is not the worst solution, even though it's nonstandard. I
strongly believe that SQL code cut/pasted into an MS Word document
will not be long-term viable, and the same goes for ``e-mail your code
to one person who will be in charge of uploading/managing it.''

\section{Timeline and goals}

\paragraph{Already built:} A database space within CDW to store lists
of adjudicated elements. A SharePoint space for phenomics core to store
some documentation. A wiki system to store further documentation, allow
for search, allow collaborative editing with easy versioning, and help
SharePoint navigation. A very basic system to store source code (happens
to be the wiki).

\paragraph{Very short term (1--2 months):} Decide on wiki vs.\ SharePoint
vs.\ a combination vs.\ some other solution.\footnote{This depends on fixing access
  to the wiki, or fixing the search function within SharePoint.} Organize the browsing
function of the wiki (or other solution) to the specs of the phenomics
core (Anne \& I discussed this in past).

\paragraph{Long term (1 year +):} Agree on and possibly train on a more
robust source code control system (or failing that, design the wiki to do
the best we can). Agree on best practices for elements needed to document
a phenotype.\footnote{who built this, when, using what process, for
  what chief purpose/study/intent, why did they make the choices they
  made, etc.} Document at least one phenotype using these practices,
to serve as an exemplar for others. Develop a system to advertise what is in the
process of being built (not just completed tasks).

\paragraph{Even longer term:} Have a system that infers a SME's internal mental
model,\footnote{e.g.\ ``serum sodium is where LabChemTestName matches
  the following 3 text strings, where Topography matches the following
  2 strings, or where LOINC is in the following set of 5 elements,
  except in the following 20 cases which have probably erroneous
  LOINCs or TestNames\ldots{}.''} from the SME answering
``keep/discard'' to CDW elements, with only slight further effort from
the SME. (Nate has some ideas about how to approach this, e.g.\ a system that learns
classification trees that the SME can select/edit at a high level.)

\paragraph{Questions for the group:} Is the wiki the right place for
SQL code? What training on the wiki do you want? What should be
captured about each algorithm?

\section{More about MediaWiki and our server}

``A wiki is a Web application which allows people to add, modify, or
delete content in collaboration with
others.''\footnote{http://en.wikipedia.org/wiki/Wiki} MediaWiki is the
software that makes Wikipedia run, but anyone can install it to their
Web server to create their own wiki. MediaWiki uses its own markup
language called wikitext. Templates avoid a \emph{lot} of copy-pasting
of wikicode. I'm more than happy to help with creating templates and dealing with markup. The first use of our wiki for lab adjudication was May 2014.

\end{document}

% LocalWords:  MAVERIC vhabosapp Jira workflow SME operationalize CDW DMR CSPCC
% LocalWords:  LabChemTestName workflows MediaWiki MySQL PHP wikitext AST HPC
% LocalWords:  GenISIS Phsieh Lalitha viswanath Mmcduffie DCP Rpourali Nilla
% LocalWords:  wikicode microvascular SME's Operationalizing SIDs SAS
%%  LocalWords:  SharePoint SharePoint's MAVIN DimLab phenomics LOINC
%%  LocalWords:  LOINCs TestNames
