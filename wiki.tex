\documentclass{tufte-handout}
\title{Recap of MAVERIC Wiki Changes}
\author{Andy Zimolzak, MD, MMSc}
\date{November 19, 2014}
% \usepackage{graphicx}
% \setkeys{Gin}{width=\linewidth,totalheight=\textheight,keepaspectratio}
%  % Gin means Graphics-INclude

\begin{document}

\maketitle

~\\

%% Lorem ipsum dolor sit amet, consectetur adipiscing
%% elit.\footnote{Nullam quis dui metus, rhoncus rhoncus purus.} 

Themes:

\begin{itemize}
\item Data Access and Secure Server Development
\item Model/algorithm development

\begin{itemize}
\item Communication
\item Documenting the process
\item Storing results
\end{itemize}

\item Model/algorithm management

\begin{itemize}
\item Storing metadata (DMR)
\item Storing models themselves
\item Model test/validation process (who signs off, on what?)
\item Storing model test and validation results
\end{itemize}
\end{itemize}

Among others, this encompasses the wiki project as well as the data
pull request tracker. These projects originated as solutions to
existing problems, and both are hopefully approaching full roll-out
phase. As such, it may be important to have defined training
objectives and roll-out plans for each tool, so that they become
integrated into workflows, rather than remain as offshoot projects.
 
Next Wednesday's meeting will be abbreviated due to the CSPCC monthly
speaker event occurring at noon (topic is fraud in medical research,
should be very interesting), so we will only have time to talk at
length about one topic.
 
Andy and Ned, if you could have a recap of the changes that have
happened with the wiki project for next week, as well as your plans
for next steps with the project, I think that would be of general
interest to the members of this meeting.
 
There has also been some feedback regarding the name of this meeting,
since it does not directly pertain to ``data.'' One name proposal is
``Model/Algorithm environments meeting'' but I don't know if that is
too vague. Thoughts?

\section{About MediaWiki}

Lorem ipsum dolor sit amet, consectetur adipiscing elit. Nullam quis
dui metus, rhoncus rhoncus purus. Mauris mauris nisl, eleifend vel
venenatis id, pellentesque sit amet nisl. Nunc vel arcu vitae neque
varius volutpat id in tortor. Nulla quam lacus, vestibulum eget
consectetur et, accumsan non turpis. Nulla facilisi. Proin ligula
nisl, luctus sit amet convallis dictum, vulputate ut arcu. Donec
tristique, nibh et egestas semper, metus lectus laoreet mi, et varius
diam metus nec lorem. Cras nisi eros, faucibus ac sagittis eget,
pellentesque ut lectus. Etiam in est lobortis nisl commodo dapibus.
Quisque quis elit ut libero placerat euismod. Proin a dapibus est.

\section{What has changed recently}

\paragraph{October 23--28} I created several standard templates that
can be used in multiple places, to avoid duplicating the same wikitext
code. These are mainly for Data Pull and NLP subsections of the wiki. 

\paragraph{October 29} Organize the Data Pull wiki page structure. 

\paragraph{October 30} Start using for ``POP Additional Documents''
and sodium.

\paragraph{November 5} Shadow wiki created.

\paragraph{November 6} Magnesium, ALT.

\paragraph{November 12} Shadow wiki becomes real wiki.

\paragraph{November 12--14} Magnesium, ALT, POP surgical procedures,
ALT, AST.

\section{Suggestions for you all in future}

Don't upload enormous things. Use talk pages. 

\section{Flaws in wiki}

Where does the actual SQL go? Here is what the current Data Pull
template looks like.

\end{document}
