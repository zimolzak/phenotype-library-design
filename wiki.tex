\documentclass{tufte-handout}
\title{Design of a VA Phenotype Library}
\author{Andy Zimolzak, MD, MMSc}
\date{February 8, 2017}

\begin{document}

\maketitle

Motivation: VA data elements are not harmonized across stations.
\emph{Phenotype} is the term we use for a set of harmonized data
elements, or a higher-order concept made of many such ``building
blocks.''\footnote{For example, 130 stations' ICD codes + labs $\to$
  harmonized ICD + labs $\to$ diabetes + neuropathy $\to$ diabetic
  neuropathy $\to$ which is a subset of diabetes with any
  microvascular complication $\to$ subset of diabetes with any
  complication at all, etc.}

\section{Our adjudication process}

For a lab: Think of search strings. Write query (SQL + SAS). Return
sheet to SME for adjudication. Return to analyst. SIDs to database.
Future: save code, documentation, and data, and metadata. We have a
request for metadata in the database. Documentation can't live there
feasibly. Code probably not either. Manually updating a revision
history of a document is at best onerous---at worst unsustainable.

\noindent \begin{tabular}{lll}
\\
\hline
 & Current & Future\\
\hline
Code & Some in SharePoint & IBM Rational Team Concert or \ldots{}?\\
Documentation & SharePoint & MediaWiki or \ldots{}?\\
Ready-to-use adjudications & MS SQL Server\\
Adjudication metadata & SharePoint & MS SQL Server?\\
\hline
\\
\end{tabular}

Future: want to be national.  Other CSP studies within Boston, albeit
less informatics heavy. Other CSP studies nationally (other
coordinating centers). Non-CSP research (each local VAMC).
Non-research. DCP talks OK to MVP. MVP talks OK to BD-STEP. But those
are relatively informal. Trying to do user contributed and comparison
with OMOP. Slow going on both fronts is my impression. Maybe due to
investment and trust in our own processes and lack of same in others'?
Code sharing and trust.

%%% note: go get OMOP. Go get DIMLAB2. 

\section{Example phenotypes}

%% table of fake sodium? fake a1c?

Failed first line platinum containing chemotherapy. DCP often
wanting to know ``where did this come from?''

\section{What's in CDW Library---and misc.}

Potassium, albumin, total cholesterol, etc. HSR-DATA Listserv is a
resource. Virec is a resource. Lab notebook metaphor. Rocket or
satellite metaphor.

\section{Demo}

Nate's thing: active learning.

\section{Future}

List of popular phenotypes and their metadata. For the very most
popular ones, counts, monthly perhaps? What data model? OMOP?
Probabilistic or complex phenotype. 

\section{Why in the first place}

The basic idea is that the methods of science should be documented
well enough to be reproducible---in other words, ``show your work.''
For our type of science, \emph{code is method.} See also
http://sciencecodemanifesto.org

``key components of publications should be available to peers who wish
to validate the techniques and results. A core element of many papers
is the computer code used by authors in models, simulations and data
analysis.\ldots{} Accordingly, our policy now mandates that when code
is central to reaching a paper's conclusions, we require a statement
describing whether that code is available and setting out any
restrictions on accessibility.''\footnote{\emph{Nature} 514:536,
  2014-10-30}

``This often necessitates that code be written quickly with no
expectation that it will be maintained or used by oth- ers in their
experiments. But that code is still an integral part of the materials
and methods. Just as with biological materials, such code should be
made available alongside results obtained using it, with sufficient
explanation as to be interpretable. This includes a minimally
acceptable level of commenting in the code, a description of software
dependencies, and basic installation and use instructions. Because
software support is far more burdensome than delivery of materials
such as plasmids, this code may be accompanied by a disclaimer that it
is released only for reporting and archival
purposes.''\footnote{\emph{Nature Methods} 11 (3): 211. March, 2014.}

Going even further, When Frank Lederle \emph{et al.}\ publish the DCP
results, will it be possible to run the whole analysis with one click?
If not, why not?

\noindent \rule{\textwidth}{1pt}

\section{What I imagine for a phenotype library}

When a subject matter expert (SME), analyst, statistician, and/or data
pull engineer create a phenotype, the result can be: SQL code that
represents the SME's rules, or a public database table that represents
the enumerated list of included/excluded data items, or even a
statistical model of the likelihood of a patient having the given
phenotype. Ideally, a robust phenotype library should exist, serving
several functions.

(1) The library would make the products of each SME/data review
process \emph{available for viewing} by study coordinators or others,
with ``provenance'' (responsible SME and data extractor) clearly
noted, to allow study staff to trust a given phenotype. It should also
make it easy to browse an overview of all phenotypes without having to
drill down into numerous different subfolders. (2) The library would
make phenotypes \emph{available for reuse}, for the purposes of other
projects. (3) The library would allow easy \emph{searches across
  phenotypes,} to allow new studies to discover what already exists.
Importantly, search should cover the contents of each document---not
just titles. (4) The library would make phenotypes \emph{available for
  renewal,} with their date of creation clearly noted, because data
warehouse elements may change over time. (5) The library would
\emph{document} the rationale for decisions made by SME, analyst,
etc.\ and allow collaborative editing of this documentation. Ideally,
changes would be audited/tracked. Possibly it should allow links among
documents. (6) The library should be available across VA nationally.

\section{Operationalizing the library}

I think a wiki serves \emph{some} of these requirements well,
especially search; and viewing provenance, documentation, and
rationale.\footnote{A wiki is a Web application that allows people to
  add, modify, or delete content in collaboration with others. Note
  that functions 1, 3, and 5 involve human-readable English text.} But
I'm not sure whether the current wiki is the best long-term solution.
Note that SharePoint has a search feature, but it currently does not
work at all.

A wiki does \emph{not} serve some of the functions as
well.\footnote{Specifically reusing, renewing, or modifying
  phenotypes. These functions depend more on machine-readable items.}
In my imagined ideal world, people would use Git or other source code
management software to store SQL, SAS, etc.\ code. The main drawback
is the learning curve, so I don't know whether this is actually going
to happen. Storing code on the wiki is not the worst solution, even
though it's nonstandard. I strongly believe that SQL code cut/pasted
into an MS Word document will not be long-term viable, and the same
goes for ``e-mail your code to one person who will be in charge of
uploading/managing it.''

\marginnote{\textbf{More about our current wiki:} It uses MediaWiki,
  which is the software that makes Wikipedia run. MediaWiki uses its
  own markup language and also provides a template system which helps
  avoid a lot of copy-pasting of markup code, and a range of other
  features. I'm more than happy to help with creating templates and
  dealing with markup. The MediaWiki instance hosted for us by VINCI
  is currently having authentication problems, however.}

\section{Timeline and goals}

\paragraph{Already built:}
A database space within CDW to store lists of adjudicated elements. A
SharePoint space for the phenomics core to store some documentation. A
wiki system to store further documentation, allow for search, allow
collaborative editing with easy versioning, and to help with
SharePoint navigation. A very basic system to store source code (this
system happens to be the wiki). Note that SharePoint has a wiki
function too; also the SharePoint search function may be improved
somewhat in summer, but I'm not sure to what extent.

\paragraph{Very short term (1--2 months):}
Decide on continuing with current wiki vs.\ SharePoint vs.\ a
combination vs.\ some other solution. Organize the browsing function
of the library to the specs of the phenomics core (Anne \& I discussed
this in the past).

\paragraph{Long term (1 year +):}
Agree on---and possibly train on---a more robust source code control
system (or failing that, design the library to do the best we can).
Agree on best practices for elements needed to document a
phenotype.\footnote{E.g.\ who built this, when, using what process,
  for what chief purpose/study/intent, why did they make the choices
  they made, etc.} Document at least one phenotype using these
practices, to serve as an exemplar for others. Develop a system to
advertise what is \emph{in the process of being built} (not just
completed tasks).

\paragraph{Even longer term:}
Have a system that infers a SME's internal mental
model,\footnote{E.g.\ ``serum sodium is where LabChemTestName matches
  these 3 given text strings, where Topography matches these 2 given
  strings, or where LOINC is one of these 5 given codes, except in
  these 20 given cases which have probably erroneous LOINCs or
  TestNames\ldots{}.''} from the SME answering ``keep/discard'' to CDW
elements, with only slight further effort from the SME.\footnote{Nate
  Fillmore (BD-STEP fellow) has some ideas about how to approach this,
  e.g.\ a system that learns classification trees that the SME can
  select/edit at a high level.}

\section{Questions for the group}

Do you agree with my list of requirements? Does a wiki fit the bill
(see ``Very short term'' goals above)? Is the wiki the right place for
SQL code? What training or further information on the wiki do you
want? What should be captured about each phenotype?

\end{document}

% LocalWords:  MAVERIC vhabosapp Jira workflow SME operationalize CDW DMR CSPCC
% LocalWords:  LabChemTestName workflows MediaWiki MySQL PHP wikitext AST HPC
% LocalWords:  GenISIS Phsieh Lalitha viswanath Mmcduffie DCP Rpourali Nilla
% LocalWords:  wikicode microvascular SME's Operationalizing SIDs SAS
%%  LocalWords:  SharePoint SharePoint's MAVIN DimLab phenomics LOINC
%%  LocalWords:  LOINCs TestNames subfolders VINCI
