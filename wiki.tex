\documentclass{tufte-handout}
\title{What Records Should VA Keep About Phenotypes?}
\author{Andrew Zimolzak, MD, MMSc}
\date{March 27, 2017}

\begin{document}

\maketitle

%% fixme
Like essentially all studies that rely heavily on the VA Corporate
Data Warehouse, MVP must deal with the fact that VA data elements are
incompletely harmonized across the 130+ VA Medical Centers or
``stations,''\footnote{``[D]ata aggregation across the VHA is highly
  problematic, and data validity is often impossible to verify.''
  \emph{N Engl J Med} 373:1693--1695} although messy data is certainly
not unique to VA.\footnote{Example hemoglobin A1c units: ``\%, \%HB,
  \% A1C, MG/DL, G/DL, NULL, Blank, U.'' \emph{Pharmacoepidemiol Drug
    Saf.} 23(6):609--18.} Even when stations are forced to use the
same coding system nationwide (e.g.\ ICD diagnosis codes), we must
perform further work to determine the reliability of any combination
of data elements. \emph{Phenotype} is the term we use for a set of
harmonized data elements, or a higher-order concept made of many such
``building blocks.'' When a subject matter expert (SME), analyst,
statistician, and/or data pull engineer create a phenotype, the result
can be: SQL code that represents the SME's rules, a public database
table that represents the enumerated list of included\slash excluded
data items, or even a statistical model (in SAS or R or other code) of
the likelihood of a patient having the given
phenotype.\footnote{Example phenotypes, which happen to be
  requirements from \emph{non-}MVP studies: Failed first-line
  platinum-containing chemotherapy. Urgent coronary revascularization
  (completed or attempted) because of unstable angina. Erectile
  dysfunction, defined as first prescription for PDE5 inhibitor or
  referral for ED.}

\section{What we can't do today}

Ideally, a robust phenotype library should exist, serving several
functions. (1) The library would make the products of each SME/data
review process \emph{available for viewing} by study coordinators,
either by searching, or by browsing an overview of all available
phenotypes; and these products should be
\emph{available for immediate use} by study analysts writing further
analytic code. (2) The \emph{code} used in the process of deriving a
phenotype should be preserved, in part because the scientific validity of a study's
conclusions rests in part on the phenotype code. (3) Phenotype and
code \emph{metadata} (such as responsible SME and data extractors, and
date of creation) should be clearly noted, to allow study staff to
trust a phenotype, and to allow phenotyping core personnel to renew
elements over time. (4) The library would \emph{document} the
rationale for decisions made by SME, analyst, etc.\ and allow
collaborative editing of this documentation. Documentation will
typically be in sentences and paragraphs.\footnote{E.g.\ ``[S]hould
  the ETL process be interrupted by any kind of database, host machine
  or connection issue\ldots{}, in order to protect data and avoid
  partial load the ETL will not re-execute itself and\ldots{}.'' Courtesy
  of Oleg Soloviev.}

\newpage

A central idea of functions (1)--(4) above is that the methods of
science should be available and documented well enough to be
reproducible---in other words, ``show your work.''\footnote{``[O]ur
  policy now mandates that when code is central to reaching a paper's
  conclusions, we require a statement describing whether that code is
  available and setting out any restrictions on accessibility.''
  \emph{Nature} 514:536, 2014-10-30} For our type of science,
\emph{code is method.}\footnote{sciencecodemanifesto.org} 

\section{Wish list (a.k.a.\ thinking big)}

I personally imagine keeping track of all analyses run on the dataset,
starting even at the early exploratory stage. I would imagine the same
record-keeping attitude that one might have toward a laboratory
notebook in a basic bioscience lab, or toward a patent
book.\footnote{``Just as with biological materials, such code should
  be made available alongside results obtained using it\ldots{}.
  Because software support is far more burdensome than delivery of
  materials such as plasmids, this code may be accompanied by a
  disclaimer\ldots{}.'' \emph{Nature Methods} 11 (3): 211. March,
  2014.} %%fixme metaphor above.

For some adjudications that SMEs perform (e.g.\ laboratory), we could
develop a system that infers\slash learns a SME's internal mental
model,\footnote{E.g.\ ``serum sodium is where LabChemTestName matches
  these 3 strings, or where LOINC is one of these 5 codes, except in
  these 20 cases\ldots{}.''} from the SME answering ``keep\slash
discard'' to database elements, with only slight further effort from
the SME. For the very most popular phenotypes, we could display
database counts, updated monthly perhaps. Phenotypes could be linked
back to specific clauses in a given version of the research study
protocol; any amendment to a critical part of the protocol would
trigger a need for an amendment to the corresponding data process.
When study results are published, will it be possible to run the whole
analysis with one click? If not, \emph{why not?}

The library should be available across VA nationally. The library
would make phenotypes \emph{available for reuse}, for the purposes of
other projects. Not only MVP, but VA Cooperative Studies Program (CSP)
trials---in Boston and at the other coordinating centers---would
benefit from simple, reusable phenotypes. Non-CSP research projects at
single VA Medical Centers would benefit, as would non-research efforts
(quality and business related).

Collaboration. DCP talks OK to MVP. MVP talks OK to BD-STEP. But those
are relatively informal. Extremely hard to sync up with people beyond
that. You're working on your thing, then you find out somebody else
has been working simultaneously, not in the exact same direction, but
in a similar one. It's an enormous proposition to brain dump from you
to them and vice versa, and then to decide on a common direction. The
sync up process can effectively halt all work at times, and both sides
are likely to see that some of their work (sunk costs) does not get
incorporated in the final product. Need to trust others' processes.
Code sharing could help but not completely.

\end{document}

% LocalWords:  MAVERIC vhabosapp Jira workflow SME operationalize CDW DMR CSPCC
% LocalWords:  LabChemTestName workflows MediaWiki MySQL PHP wikitext AST HPC
% LocalWords:  GenISIS Phsieh Lalitha viswanath Mmcduffie DCP Rpourali Nilla BD
% LocalWords:  wikicode microvascular SME's Operationalizing SIDs SAS VHA Engl
%%  LocalWords:  SharePoint SharePoint's MAVIN DimLab phenomics LOINC Giroir dl
%%  LocalWords:  LOINCs TestNames subfolders VINCI incentivized Wilensky lll HB
% LocalWords:  CSP VAMC OMOP hould ETL llllllll lllll Hgb Glyco Retic Raebel ur
% LocalWords:  Pharmacoepidemiol Saf glycosylated HSR Listserv Virec plasmids
% LocalWords:  Timeline versioning revascularization PDE ata Oleg Soloviev
% LocalWords:  sciencecodemanifesto Lederle
