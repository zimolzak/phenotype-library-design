\documentclass{tufte-handout}
\title{Recap of MAVERIC Wiki Changes}
\author{Andy Zimolzak, MD, MMSc}
\date{November 19, 2014}

\begin{document}

\maketitle

\section{Problem/niche}

The Precision Oncology Project and the Point of Care Trial project
rely in part on a workflow that goes roughly as follows. \emph{First,}
the subject matter expert knows that we want to track the effect of
drug $x$ on (for example) serum sodium. The SME attempts to
operationalize the concept of serum sodium, suggesting a list of terms
to be searched for in a field (e.g.\ \texttt{LabChemTestName}) of CDW.
\emph{Second,} the data pull team returns a list of lab tests matching
these search terms.\footnote{The list sent to the SME is often an
  Excel file. Data extractor probably generates SQL code in the
  process.} \emph{Third,} the SME reviews the list of tests and
decides (adjudicates) which items truly match the concept. The SME
returns either a refined general rule to specify which items match the
concept, or an enumerated list of which items match the
concept.\footnote{So far I return this adjudication to Beth in the
  form of an edited Excel file.} \emph{Fourth,} the data pull team
creates a query that represents this adjudication.\footnote{probably
  generating refined SQL code.} The clinical SME may give formal
approval or ``sign off'' on the data pull at step 3, or possibly on
the results of step 4.

The data pull and NLP teams intend to use the wiki to capture this
workflow so that these deliverables are \emph{available for viewing}
by study coordinators or others, with ``provenance'' (responsible SME
and data extractor) clearly noted; \emph{available for reuse} in a
library, for the purposes of other projects; and \emph{available for
  renewal,} with their date of creation clearly noted, because CDW
elements may change over time. The rationale for the SME
including/excluding certain tests can also be documented.\footnote{The
  basic idea is that the methods of science should be documented well
  enough to be reproducible---in other words, ``show your work.'' For
  our type of science, \emph{code is method.} See also
  http://sciencecodemanifesto.org}

\section{Recent wiki changes} 

Started to use as of May 2014 on vhabosapp8. adjudications as of oct
30, starting with radiology note titles, sodium, magnesium, AST, ALT,
POP surgical procedures.

Templates avoid a \emph{lot} of copy-pasting of wikicode.

\section{Suggestions for you all in future}

Look at the wiki. Make a user account. Edit things (be bold). Ask if
you need help.

\section{Questions for the group}

\begin{itemize}

\item What should we call the deliverable that the data pull expert
  and the SME create in collaboration?\footnote{Thoughts: data
    model, concept model, algorithm, adjudication, library, phenotype,
    phenotype model, study data dictionary.}

\item Not much complete SQL is on the wiki currently. Is the wiki the
  right place for SQL code? Or should the wiki merely have a pointer
  to a source control repository?

\item What training on the wiki do you want?

\item Do you want an e-mail every time a new algorithm is uploaded?

\item What should be captured about each algorithm?\footnote{Currently
  caputured on basic algorithm pages on wiki: algorithm version, SQL
  statement, SME name, approval date. On the more detailed wiki form
  we have the following fields: lab name, sample source (body site),
  date created, data source, SQL search terms used, clinical
  adjudication list (link to Excel file), adjudicator name, SQL data
  extractor name, extracted data location, content of extracted data,
  extraction verified by:, SQL extraction code.}

\end{itemize}

\section{Further reading about MediaWiki and our server}

``A wiki is a Web application which allows people to add, modify, or
delete content in collaboration with
others.''\footnote{http://en.wikipedia.org/wiki/Wiki} MediaWiki is the
software that makes Wikipedia run, but anyone can install it to their
Web server to create their own wiki. MediaWiki uses its own markup
language called wikitext.

\end{document}

% LocalWords:  MAVERIC vhabosapp Jira workflow SME operationalize CDW DMR CSPCC
% LocalWords:  LabChemTestName workflows MediaWiki MySQL PHP wikitext AST HPC
% LocalWords:  GenISIS Phsieh Lalitha viswanath Mmcduffie DCP Rpourali Nilla
% LocalWords:  wikicode
