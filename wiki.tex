\documentclass{tufte-handout}
\title{Design of a VA Phenotype Library}
\author{Andrew Zimolzak, MD, MMSc}
\date{March 27, 2017}

\begin{document}

\maketitle

It all starts with the research protocol! What should we capture and
why? VA data elements are not harmonized across stations. We have 130+
stations which are not incentivized to represent data in the same
way.\footnote{``[D]ata aggregation across the VHA is highly problematic,
  and data validity is often impossible to verify.'' Giroir BP,
  Wilensky GR, \emph{N Engl J Med} 373:1693--1695. PMID: 26422644}
Ugly units in lab tests are not unique to VA,
however.\footnote{Examples for hemoglobin A1c lab: ``\%, \%HB, \% A1C,
  MG/DL, G/DL, NULL, Blank, U.'' Raebel \emph{et al. Pharmacoepidemiol
    Drug Saf.} 2014 Jun;23(6):609--18.} \emph{Phenotype} is the term
we use for a set of harmonized data elements, or a higher-order
concept made of many such ``building blocks.''\footnote{Other terms:
  data cleaning, data characterization, data integration, data
  modeling, data aggregation} DCP often wanting to know ``where did
this come from?''\footnote{Failed first line platinum containing
  chemotherapy. Urgent coronary revascularization (completed or
  attempted) because of unstable angina. Hospitalization for acute
  congestive heart failure. Erectile dysfunction, defined as first
  prescription for PDE5 inhibitor or referral for ED. Doubling of
  serum creatinine from baseline. First use of a medication for
  diabetes.}

\section{List of wants}

Code versus documentation versus data versus
metadata.\footnote{Example DCP documentation: ``[S]hould the ETL
  process be interrupted by any kind of database, host machine or
  connection issue\ldots{}, in order to protect data and avoid partial
  load the ETL will not re-execute itself and a warning message will
  be written into the log table on the next execution
  attempts\ldots{}.''}

Future: want to be national. Other CSP studies within Boston, albeit
less informatics heavy. Other CSP studies nationally (other
coordinating centers). Non-CSP research (each local VAMC).
Non-research. DCP talks OK to MVP. MVP talks OK to BD-STEP. But those
are relatively informal. Trying to do user contributed and comparison
with OMOP. Slow going on both fronts is my impression. Maybe due to
investment and trust in our own processes and lack of same in others'?
Code sharing and trust.\footnote{``Just as with biological materials,
  such code should be made available alongside results obtained using
  it\ldots{}. Because software support is far more burdensome than
  delivery of materials such as plasmids, this code may be accompanied
  by a disclaimer\ldots{}.'' \emph{Nature Methods} 11 (3): 211. March,
  2014.}

%%% note: go get OMOP. Go get DIMLAB2. Get list of sta3ns?

List of popular phenotypes and their metadata. For the very most
popular ones, counts, monthly perhaps? What data model? OMOP?
Probabilistic or complex phenotype.

Have a system that infers a SME's internal mental
model,\footnote{E.g.\ ``serum sodium is where LabChemTestName matches
  these 3 given text strings, where Topography matches these 2 given
  strings, or where LOINC is one of these 5 given codes, except in
  these 20 given cases which have probably erroneous LOINCs or
  TestNames\ldots{}.''} from the SME answering ``keep/discard'' to CDW
elements, with only slight further effort from the SME.

The basic idea is that the methods of science should be documented
well enough to be reproducible---in other words, ``show your
work.''\footnote{``[O]ur policy now mandates that when code is central
  to reaching a paper's conclusions, we require a statement describing
  whether that code is available and setting out any restrictions on
  accessibility.'' \emph{Nature} 514:536, 2014-10-30} For our type of
science, \emph{code is method.}\footnote{sciencecodemanifesto.org} Lab
notebook metaphor. Rocket or satellite metaphor. Going even further,
When Frank Lederle \emph{et al.}\ publish the DCP results, will it be
possible to run the whole analysis with one click? If not, why not?


When a subject matter expert (SME), analyst, statistician, and/or data
pull engineer create a phenotype, the result can be: SQL code that
represents the SME's rules, or a public database table that represents
the enumerated list of included/excluded data items, or even a
statistical model of the likelihood of a patient having the given
phenotype. Ideally, a robust phenotype library should exist, serving
several functions.

(1) The library would make the products of each SME/data review
process \emph{available for viewing} by study coordinators or others,
with ``provenance'' (responsible SME and data extractor) clearly
noted, to allow study staff to trust a given phenotype. It should also
make it easy to browse an overview of all phenotypes without having to
drill down into numerous different subfolders. (2) The library would
make phenotypes \emph{available for reuse}, for the purposes of other
projects. (3) The library would allow easy \emph{searches across
  phenotypes,} to allow new studies to discover what already exists.
Importantly, search should cover the contents of each document---not
just titles. (4) The library would make phenotypes \emph{available for
  renewal,} with their date of creation clearly noted, because data
warehouse elements may change over time. (5) The library would
\emph{document} the rationale for decisions made by SME, analyst,
etc.\ and allow collaborative editing of this documentation. Ideally,
changes would be audited/tracked. Possibly it should allow links among
documents. (6) The library should be available across VA nationally.

\section{Personal note}

Extremely hard to sync up with people. You're working on your thing,
then you find out somebody else has been working simultaneously, not
in the exact same direction, but in a similar one. It's an enormous
proposition to brain dump from you to them and vice versa, and then to
decide on a common direction. The sync up process can effectively halt
all work at times, and both sides are likely to see that some of their
work (sunk costs) does not get incorporated in the final product.


\end{document}

% LocalWords:  MAVERIC vhabosapp Jira workflow SME operationalize CDW DMR CSPCC
% LocalWords:  LabChemTestName workflows MediaWiki MySQL PHP wikitext AST HPC
% LocalWords:  GenISIS Phsieh Lalitha viswanath Mmcduffie DCP Rpourali Nilla BD
% LocalWords:  wikicode microvascular SME's Operationalizing SIDs SAS VHA Engl
%%  LocalWords:  SharePoint SharePoint's MAVIN DimLab phenomics LOINC Giroir dl
%%  LocalWords:  LOINCs TestNames subfolders VINCI incentivized Wilensky lll HB
% LocalWords:  CSP VAMC OMOP hould ETL llllllll lllll Hgb Glyco Retic Raebel
% LocalWords:  Pharmacoepidemiol Saf glycosylated HSR Listserv Virec plasmids
% LocalWords:  Timeline versioning revascularization PDE
